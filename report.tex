\documentclass[a4,10pt,twoside]{report}

\usepackage{amsmath}

\begin{document}
	\title{Summer Project Notes}
	\date{9 June 2016}
	\author{Sohom Roy}
	\maketitle
	
	
	\tableofcontents

	
	\chapter{First Chapter}
	\section{The equations governing the expansion of the universe}
	
	In cosmology, we come across three important equations which are central to our understanding of how the universe evolves with time.\\ \\
	The first of these is the Friedmann equation, which relates the scale factor $a(t)$, the curvature $\kappa$ of the universe, the energy density, $\varepsilon$ and tells us how the scale factor evolves with time.\\
	
	The equation is as given below:
	\begin{equation}
	\left(\frac{\dot{a}}{a}\right)^2=\frac{8\pi G}{3c^2}\varepsilon-\frac{\kappa c^2}{R_0^2a^2}
	\end{equation}
	
	$\kappa$ can be either equal to $0$, $+1$, $-1$, corresponding respectively to a flat, positively curved and a negatively curved universe.
	The left hand side is the square of the Hubble constant $H(t)$.
	
	The second equation is the fluid equation, which is derived from thermodynamics, and is given by
	
	\begin{equation}
		\dot{\varepsilon}+3\frac{\dot{a}}{a}(\varepsilon+P)=0
	\end{equation}
	
	The third equation is the acceleration equation, which is not really an independent equation, but is derived from the above two equations.
	
	\begin{equation}
		\frac{\ddot{a}}{a}=-\frac{4\pi G}{3c^2}(\varepsilon+3P)
	\end{equation}
	
	The last equation is the equation of state, which gives a relation between the pressure $P$ and energy density $\varepsilon$.
	
	\begin{equation}
		P=w\varepsilon
	\end{equation}
	
	For non-relativistic matter, $w=0$. For relativistic matter(radiation), $w=\frac{1}{3}$.
	
	There is another component of the universe, the cosmological constant, $\Lambda$, which was introduced by Einstein to account for a static universe containing only matter and radiation. It has $w=-1$.
	
	In the Friedmann equation, putting the curvature constant $\kappa$ equal to zero, gives us a value for the energy density, which we call the critical density.
	
	\begin{equation}
	H^2(t)=\frac{8\pi G}{3c^2}\varepsilon_c
	
	\end{equation}
	
\end{document}