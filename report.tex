\documentclass[a4,12pt,oneside]{report}

\usepackage{amsmath}
\usepackage{amssymb}
\usepackage{geometry}\geometry{left = 1in, right = 1in, top = 1in, bottom = 1in}
\usepackage{hyperref}


\begin{document}
	\title{Summer Project Notes}
	\date{9 June 2016}
	\author{Sohom Roy}
	\maketitle
	
	
	\tableofcontents

	
	\chapter{First Chapter}
	
	\section{Form of the metric for a homogeneous, isotropic universe}
	
	Initially, Newton had assumed that space is Euclidean. Then the distance between two points assumes the very simple form
	
	\begin{equation}
		ds^2=dx^2+dy^2+dz^2
	\end{equation}
	
	if we are using a Cartesian coordinate system. In spherical coordinates, this can be expressed as

	\begin{align}
	ds^2=dr^2+ r^2d\theta^2 + r^2sin^2\theta d\phi^2	\\
	\Rightarrow ds^2=dr^2+ r^2(d\theta^2+sin^2\theta d\phi^2) \\
	\Rightarrow ds^2=dr^2+ r^2d\Omega^2 
	\end{align}
	
	But now, from Einstein's theory of general relativity we see that space can have curvature. That is, space can either be positively curved, flat(Euclidean), or negatively curved.
	
	The three possible metrics for a homogeneous, isotropic, three-dimensional space can be compactly written down as
	
	\begin{equation}
	    ds^2=dr^2+S_\kappa(r)^2d\Omega^2
	\end{equation}
	
	where 
	
	\begin{equation}
		d\Omega^2=d\theta^2+sin^2\theta d\phi^2
	\end{equation}
	
	and 
	\begin{equation}
		S_\kappa(r)=\begin{cases}R\:\sin\left(\dfrac{r}{R}\right),   &(\kappa=+1) \\
		r,  &(\kappa=0) \\
		R\:\sinh\left(\dfrac{r}{R}\right),  &(\kappa=-1)
		\end{cases}
	\end{equation}
	
	Thw Minkowski metric is given by
	
	\begin{equation}
		ds^2=-c^2dt^2+dr^2+r^2d\Omega^2
	\end{equation}
	
	Thsi metric applies for a flat or Euclidean space-time. This metric deals with the special case when space-time is not curved by the presence of mass and energy.
	
	\subsection{Robertson-Walker metric}
	
	This metric was introduced to describe a univere which was spatially homogeneous and isotropic, and where distances are allowed to expand or contract as a function of time. It is given by
	
	\begin{equation}
		ds^2=-c^2dt^2+a(t)^2\left[\frac{dx^2}{1-\dfrac{\kappa x^2}{R_0^2}} +x^2d\Omega^2\right]
	\end{equation}
	
	Here $a(t)$ is the scale factor, and the expression in square brackets is the metric for a uniformly curved space of radius $R_0$.

	This equation can be rewritten as 
	\begin{equation}
		ds^2=-c^2dt^2+a(t)^2[dr^2+S_\kappa(r)^2d\Omega^2]
	\end{equation}
	
	$ (r,\theta, \phi) $ are called the comoving coordinates. If the expansion of the universe is perfectly homogeneous and isotropic, then the comoving coordinates are constant with time. \\
	
	The scale factor $ a(t) $ is normalized so that $ a(t_0) =1$.
	
	\section{Proper distance}
	The distance to a galaxy situated at a comoving coordinate $ (r,\theta,\phi) $ is given by the length of the spatial geodesic between these points when the scale factor is fixed at $ a(t) $. \\
	
	So, the distance is given by the Robertson-Walker metric at a fixed time $ t $.
	\begin{equation}
		ds^2=a(t)^2[dr^2+S_\kappa(r)^2d\Omega^2]
	\end{equation}
	
	Along the spatial geodesic joining the observer and the galaxy, $ \theta $ and $ \phi $ are constant. So, the metric becomes
	
	\begin{align}
		ds^2=a(t)^2dr^2 \\
		\Rightarrow ds=a(t)dr \\
	\end{align}
	$\therefore$ Proper distance =\[d_p(t)=a(t)\int_0^r dr= a(t).r  \]
	
	Rate of change of proper distance:
	\begin{align}
		\dot{d_p}=\dot{a} r = \frac{\dot{a}}{a}d_p \\
		\Rightarrow \frac{\dot{d_p}}{d_p}=\frac{\dot{a}}{a}
	\end{align}
	
	So, we can see that there is a linear relation between the proper distance to a galaxy and its recession speed:
	
	\begin{equation}
		v_p(t_0)=H_0d_p(t_0)
	\end{equation}
	
	where $v_p(t_0)=\dot{d_p}(t_0)$ and $ H_0=\left(\dfrac{\dot{a}}{a}\right)_{t=t_0} $. \\
	
	
	The angular position of a galaxy can easily be known, but not its proper distance, or its comoving distance $ r $. \\
	
	Light emitted from the galaxy at time $ t_e $ is observed by us at a time $ t_0 $. During its travel, light travelled along a null geodesic($ ds=0 $). We have $ \theta $ and $ \phi $ constant.
	
	\begin{align}
		c^2dt^2=a(t)^2dr^2 \\
		\Rightarrow c\frac{dt}{a(t)}=dr
	\end{align}
	
	For a single wave crest emitted at time $ t_e $ and observed at time $ t_0 $, we have
	
	\begin{equation}
		c\int_{t_e}^{t_0} \frac{dt}{a(t)} = \int_0^r dr = r
	\end{equation}
	
	If wavelength of emitted light is given by $ \lambda_e $ and that of the observed light is given by $ \lambda_o $ (in general, $ \lambda_o \neq \lambda_e $), then we can say that the next wave crest of light is emitted at time $ t_e+\dfrac{\lambda_e}{c} $ and observed at time $ t_o + \dfrac{\lambda_o}{c} $.
	
	Thus, we can write
	
	\begin{equation}	
		c\int_{t_e+\frac{\lambda_e}{c}}^{t_o+\frac{\lambda_o}{c}}= \int_0^r dr = r.
	\end{equation}
	
	Comparing these equations, we see that
	
	\begin{equation}
		\int_{t_e}^{t_0} \frac{dt}{a(t)} = \int_{t_e+\frac{\lambda_e}{c}}^{t_o+\frac{\lambda_o}{c}} \frac{dt}{a(t)}
	\end{equation}
	
	Subtracting $ \int_{t_e+\frac{\lambda_e}{c}}^{t_0} \frac{dt}{a(t)} $ from both sides, we get
	
	\begin{equation}
		\int_{t_e}^{t_e+\frac{\lambda_e}{c}} \frac{dt}{a(t)} = \int_{t_0}^{t_o+\frac{\lambda_o}{c}} \frac{dt}{a(t)}
	\end{equation}
	
	We can also consider the scale factor to remain constant, since the universe does not expand by a significant amount during the time between two crests of the wave. \\
	
	So, we have
	
	\begin{align}
		&\frac{1}{a(t_e)}\int_{t_e}^{t_e+\frac{\lambda_e}{c}} dt = \frac{1}{a(t_o)}\int_{t_o}^{t_o+\frac{\lambda_o}{c}} dt	
		\\
		&\Rightarrow \frac{1}{a(t_e)}.\frac{\lambda_e}{c} =\frac{1}{a(t_o)}.\frac{\lambda_o}{c}
		\\
		&\Rightarrow \frac{1}{a(t_e)}.\lambda_e = \frac{1}{a(t_o)}.\lambda_o
		\\
		&\Rightarrow \frac{\lambda_o}{\lambda_e} = \frac{a(t_o)}{a(t_e)}
		\\
		&\text{Now,} z=\frac{\lambda_o-\lambda_e}{\lambda_e} \\
		&\Rightarrow z=\frac{\lambda_o}{\lambda_e}-1 \\
		&\Rightarrow 1+z = \frac{\lambda_o}{\lambda_e} = \frac{a(t_o)}{a(t_e)} = \frac{1}{a(t_e)}
	\end{align}
	
	
	\section{The equations governing the expansion of the universe}
	
	In cosmology, we come across three important equations which are central to our understanding of how the universe evolves with time.\\ \\
	The first of these is the Friedmann equation, which relates the scale factor $a(t)$, the curvature $\kappa$ of the universe, the energy density, $\varepsilon$ and tells us how the scale factor evolves with time.\\
	
	The equation is as given below:
	\begin{equation}
	\left(\frac{\dot{a}}{a}\right)^2=\frac{8\pi G}{3c^2}\varepsilon-\frac{\kappa c^2}{R_0^2a^2}
	\end{equation}
	
	$\kappa$ can be either equal to $0$, $+1$, $-1$, corresponding respectively to a flat, positively curved and a negatively curved universe.
	The left hand side is the square of the Hubble constant $H(t)$.
	
	The second equation is the fluid equation, which is derived from thermodynamics, and is given by
	
	\begin{equation}
		\dot{\varepsilon}+3\frac{\dot{a}}{a}(\varepsilon+P)=0
	\end{equation}
	
	The third equation is the acceleration equation, which is not really an independent equation, but is derived from the above two equations.
	
	\begin{equation}
		\frac{\ddot{a}}{a}=-\frac{4\pi G}{3c^2}(\varepsilon+3P)
	\end{equation}
	
	The last equation is the equation of state, which gives a relation between the pressure $P$ and energy density $\varepsilon$.
	
	\begin{equation}
		P=w\varepsilon
	\end{equation}
	
	For non-relativistic matter, $w=0$. For relativistic matter(radiation), $w=\frac{1}{3}$.
	
	There is another component of the universe, the cosmological constant, $\Lambda$, which was introduced by Einstein to account for a static universe containing only matter and radiation. It has $w=-1$.
	
	In the Friedmann equation, putting the curvature constant $\kappa$ equal to zero, gives us a value for the energy density, which we call the critical density.
	
	\begin{align}
	H^2(t)=\frac{8\pi G}{3c^2}\varepsilon_c \\
	\Rightarrow \varepsilon_c=\frac{3c^2}{8\pi G}H^2(t)
	\end{align}
	
	If we evaluate this at the current time, then we have
	
	\begin{equation}
		\varepsilon_{c,0}=\frac{3c^2H_0^2}{8\pi G}
	\end{equation}
    
    If energy density is greater than this, then universe is positively curved, else, it is negatively curved. 
    \\ \\
    The currently estimated value of the critical density is 
    
    \begin{equation}
	    \varepsilon_{c,0} = 5200 \pm 1000 MeV m^{-3}
    \end{equation}
    
    The equivalent mass density is given by
    
    \begin{equation}
	    \rho_{c,0}= \frac{\varepsilon_{c,0}}{c^2} = (1.1 \pm 0.3) \times 10^{11} M_\odot Mpc^{-3}
    \end{equation}
    
    We define the density parameter to be 
    
   \begin{equation}
	   \Omega \equiv \frac{\varepsilon(t)}{\varepsilon_c(t)}
   \end{equation}
   \\
  
   In terms of the density parameter, the Friedmann equation becomes
   
   \begin{equation}
	   1-\Omega(t) = -\frac{\kappa c^2}{R_0^2 a(t)^2 H(t)^2}
   \end{equation}
   
   \chapter{Second Chapter}
   
   So far, we have discussed the basic quantities and equations describing the universe, and the relations between them.
   
   In a universe with more than one component, the energy density in the equations need to be replaced by the total energy density, which is given by:
   
   \begin{equation}
	   \varepsilon_{total} = \sum_{w} \varepsilon_w
   \end{equation}
   
   and the total pressure needs to be replaced by the total pressure, given by:
   
   \begin{equation}
	   P_{total}=\sum_{w} P_w = \sum_{w} w\varepsilon_w
   \end{equation}
   
   For a single component with equation-of-state parameter w, we have
   
   \begin{align}
	   &\dot{\varepsilon}_w + 3\frac{\dot{a}}{a}(\varepsilon_w + P_w) = 0 \\
	   &\Rightarrow \dot{\varepsilon}_w + 3\frac{\dot{a}}{a}(1+w)\varepsilon_w = 0 \\
	   &\Rightarrow \dot{\varepsilon}_w = -3\frac{\dot{a}}{a}(1+w)\varepsilon_w \\
	   &\Rightarrow \int \frac{d\varepsilon_w}{w} = -3(1+w)\int \frac{da}{a} \\
	   &\Rightarrow \ln \varepsilon_w = 3(1+w) \ln a + \ln \varepsilon_{w,0} \\
	   &\Rightarrow \varepsilon_w = \varepsilon_{w,0} a^{-3(1+w)} 
   \end{align}
   
   For non-relativistic matter, $ w=0 $.
   \begin{equation}\label{key}
	   \therefore \boxed{\varepsilon_m(a) = \frac{\varepsilon_{m,0}}{a^3}}
   \end{equation}
   and for radiation, $ w=\dfrac{1}{3} $.
   \begin{equation}\label{key}
	   \therefore \boxed{\varepsilon_r(a) = \frac{\varepsilon_{r,0}}{a^4}}
   \end{equation}
   
   \chapter{Cosmological perturbation theory}
 
   \section{Perturbed fluid equations}
   Let us consider a non-relativistic fluid with mass density $ \rho $, pressure $ P <<\rho $ and velocity $ \vec{u} $, and denote its position vector by $ \vec{r} $ and time by t.
   \\
   Then the fluid equations are
   \\ \\
   {\bf\underline{Continuity Equation}:}
   \begin{equation}
   \partial_t \rho = -\overrightarrow{\nabla_r}.(\rho \vec{u})
   \end{equation}
   \\
   {\bf\underline{Euler's equation}:}
   \begin{equation}
   (\partial_t + \vec{u}.\overrightarrow{\nabla_r})\vec{u} = - \frac{\overrightarrow{\nabla_r}P}{\rho} - \overrightarrow{\nabla_r}\Phi
   \end{equation}
   
   The gravitational potential is determined by {\bf Poisson's equation}
   \begin{equation}\label{key}
   \overrightarrow{\nabla_r}^2\Phi = 4\pi G\rho
   \end{equation}
   
   Now, we introduce perturbations in the various quantities by decomposing each quantity into background quantities and perturbations. For example,
   
   \begin{equation}\label{key}
   \rho(t,\vec{r}) = \overline{\rho}(t) + \delta\rho(t,\vec{r})
   \end{equation}
   
   \section{Static space without gravity}
	In this case we consider the potential to be zero, i.e. $ \Phi = 0 $.
	
	A solution for this case is given by
	\begin{equation}\label{key}
	\overline{\rho}=\text{constant,} \: \overline{P} = \text{constant, and} \: \overline{\vec{u}} = 0.
	\end{equation} 
	
	The linearised evolution equations for the fluctuations are then given by
	\begin{align}
	\partial_t \delta\rho = - \overrightarrow{\nabla_r}.(\overline{\rho}\vec{u}) \label{eqn:6}
	\\
	\overline{\rho}\delta_t \vec{u} = - \overrightarrow{\nabla_r}\delta P \label{eqn: 7}
	\end{align}
	
	From \eqref{eqn:6} and \eqref{eqn: 7}, we get	
	\begin{equation}\label{key}
	\partial_t^2 \delta \rho - \nabla_r^2 \delta P = 0
	\end{equation}
	
	For adiabatic fluctuations, $ \delta P = c_s^2 \delta \rho $, where $ c_s$ is the speed of sound.
	\begin{equation}\label{key}
	(\partial_t^2 - c_s^2\nabla^2) \delta \rho = 0.
	\end{equation}
	
	The solution is given by a plane wave:
	\begin{equation}\label{key}
	\delta \rho = A \exp i(\omega t - \vec{k}.\vec{r})
	\end{equation}
	
	where $ \omega = c_s k  $.
	\begin{center}
	\fbox{Fluctuations oscillate with constant amplitude.}
	\end{center}
	
	\section{Static space with gravity}
	This case is similar to the last one, but we have a potential source here, which needs to be included. 
	
	\begin{equation}\label{key}
	(\partial_t^2 - c_s^2 \nabla_r^2)\delta \rho = 4 \pi G \delta \rho
	\end{equation}
	
	where we have used the perturbed Poisson equation
	
	\begin{equation}\label{key}
	\nabla^2 \delta \Phi = 4 \pi G \delta \rho
	\end{equation}
	
	This is still solved by 
	
	\begin{equation}\label{key}
	\delta \rho = A \exp i(\omega t - \vec{k}.\vec{r})
	\end{equation}
	but with
	\begin{equation}\label{key}
	\omega^2 = c_s^2 k ^2 - 4 \pi G \overline{\rho}
	\end{equation}
	
	The critical wavenumber is found by setting $ \omega = 0 $.
	\begin{equation}\label{key}
	k_J = \frac{\sqrt{4 \pi G \overline{\rho}}}{c_s}
	\end{equation}
	
	\fbox{For small scales, pressure dominates and the oscillations occur as before.}
	\\\\
	\fbox{For large scales, gravity dominates and the fluctuations increase exponentially with time.}
   	\\\\	
	The crossover happens at the {\bf Jeans' length}
	
	\begin{equation}
	\boxed{\lambda_J = \frac{2 \pi}{k_J} = c_s\sqrt{\frac{\pi}{G\overline{\rho}}}}
	\end{equation}
	
	
	\section{Expanding space}
	
	Relationship between physical coordinates $\vec{r}$ and comoving coordinates $ \vec{x} $	is given by,
	
	\begin{equation}\label{key}
	\vec{r}(t) = a(t)\vec{x}
	\end{equation}
	
	The velocity is given by
	\begin{equation}\label{key}
	\vec{u}(t) = \dot{\vec{r}} = \dot{a}\vec{x} + a\dot{\vec{x}} = \frac{\dot{a}}{a}.a(t)\vec{x} + a\dot{\vec{x}}
	\end{equation}
	\begin{equation}\label{key}
	=H\vec{r} + a\dot{\vec{x}}
	\end{equation}
	
	In the static case, the time and space derivatives defined from $ t $ and $ \vec{r} $ were independent. But this is not the case anymore.
	
	It is convenient to use comoving coordinates $ \vec{x} $ instead.
	Using the relation, 
	\begin{equation}\label{key}
	\vec{r}(t)=a(t)\vec{x} \text{  ,}
	\end{equation}
	we can write 
	\begin{equation}\label{key}
	\overrightarrow{\nabla_r} = a^{-1}\overrightarrow{\nabla_x}
	\end{equation}
	
	Relationship between time derivatives at fixed $ \vec{r} $ and fixed $ \vec{x} $ is,
	\begin{align}\label{key}
	\left(\frac{\partial}{\partial t}\right)_r &= \left(\frac{\partial}{\partial t}\right)_x + \left(\frac{\partial \vec{x}}{\partial t}\right)_r.\overrightarrow{\nabla_x} \nonumber
	\\[12pt]
	&=\left(\frac{\partial}{\partial t}\right)_x+\left(\frac{\partial a^{-1}\vec{r}}{\partial t}\right)_r.\overrightarrow{\nabla_x} \nonumber
	\\
	&=\left(\frac{\partial}{\partial t}\right)_x - \frac{\dot{a}}{a^2}.\vec{r}.\overrightarrow{\nabla_x} \nonumber
	\\
	&=\left(\frac{\partial}{\partial t}\right)_x - H\vec{x}.\overrightarrow{\nabla_x}
	\end{align}
	
	Substituting these quantities into the continuity equation, we get \\ \\
	{\bf \underline{Continuity equation:}}
	\begin{equation}\label{key}
	\left[\frac{\partial}{\partial t} - H\vec{x}.\overrightarrow{\nabla}\right].[\overline{\rho}(1+\delta)] + \frac{1}{a}.\overrightarrow{\nabla}.[\overline{\rho}(1+\delta)(Ha\vec{x}+\vec{v})] = 0
	\end{equation}
	
	\begin{center}
		Here $ \delta \equiv \dfrac{\delta\rho}{\overline{\rho}} $ is the {\bf density contrast}.
	\end{center}
	
	For the zeroth order in fluctuations,(dropping perturbations $ \delta $ and $ \vec{v} $), we have
	
	\begin{equation}\label{eqn: 24}
	\frac{\partial \overline{\rho}}{\partial t} + 3H\overline{\rho} = 0         \quad\quad (\because \overrightarrow{\nabla_x}.\vec{x} = 3)
	\end{equation}
	
	For the first order in fluctuations, we have
	\begin{align}\label{key}
	&\left[\frac{\partial}{\partial t} - H\vec{x}.\overrightarrow{\nabla}\right][\overline{\rho}\delta] + \frac{1}{a}\overrightarrow{\nabla}.[\overline{\rho}Ha\vec{x}\delta + \overline{\rho}\vec{v}] = 0 \nonumber
	\\
	&\Rightarrow\delta\frac{\partial \overline{\rho}}{\partial t} + \overline{\rho}\frac{\partial \delta}{\partial t} - \overline{\rho}H\vec{x}.\overrightarrow{\nabla}\delta + \overline H.3.\delta + \overline{\rho} H \vec{x}. \overrightarrow{\nabla} \delta + \frac{\overline{\rho}}{a}\overrightarrow{\nabla}.\vec{v} = 0 \nonumber
	\\
	&\Rightarrow\left(\frac{\partial \overline{\rho}}{\partial t} + 3H\overline{\rho}\right)\delta + \overline{\rho}\frac{\partial \delta}{\partial t} + \frac{\overline{\rho}}{a}\vec{v} = 0 \nonumber
	\end{align}
	From the zeroth order in fluctuations\eqref{eqn: 24}, we know that the coefficient of $ \delta $ is zero. 
	\begin{equation}\label{continuity}
	\frac{\partial \delta}{\partial t} = -\frac{1}{a}\overrightarrow{\nabla}.\vec{v} \Rightarrow \dot{\delta} = -\frac{1}{a}\overrightarrow{\nabla}.\vec{v}
	\end{equation}
	{\bf \underline{Euler equation:}}
	\\ \\
	\begin{equation}\label{euler}
	\dot{\vec{v}} + H\vec{v} = -\frac{1}{a\overline{\rho}}\overrightarrow{\nabla}\delta P - \frac{1}{a}\overrightarrow{\nabla}\delta\Phi
	\end{equation}
	In the absence of gravitational and pressure perturbations, we simply get $ \vec{v}\propto a^{-1} $.
	\\ \\ {\bf \underline{Poisson's equation:}}
	\begin{equation}\label{poisson}
	\overrightarrow{\nabla}^2\delta \Phi = 4\pi Ga^2\overline{\rho}\delta
	\end{equation}
	
	$ \partial_t $.\eqref{continuity} gives
	\begin{align}\label{key}
	&\frac{\partial}{\partial t}.\frac{\partial \delta}{\partial t} = \frac{\partial}{\partial t}.\left(-\frac{1}{a}\overrightarrow{\nabla}.\vec{v}\right)	
	\\
	&\Rightarrow \frac{\partial^2\delta}{\partial t^2} = -\frac{1}{a}.\frac{\partial}{\partial t}(\overrightarrow{\nabla}.\vec{v})
	\\
	&\Rightarrow\overrightarrow{\nabla}.\dot{\vec{v}} = -a\ddot{\delta}  
	\end{align}
	
	$ \overrightarrow{\nabla} $.\eqref{euler} gives 
	\begin{equation}\label{key}
	\overrightarrow{\nabla}.\dot{\vec{v}} + H\overrightarrow{\nabla}.\vec{v} = -\frac{1}{a\overline{\rho}}\overrightarrow{\nabla}^2\delta P - \frac{1}{a}\overrightarrow{\nabla}^2\delta\Phi
	\end{equation}
	
	Combining \eqref{continuity}, \eqref{euler}, \eqref{poisson}, we get
	\begin{align}
	&-a\ddot{\delta} + H\overrightarrow{\nabla}.\vec{v} = -\frac{1}{a\overline{\rho}}\overrightarrow{\nabla}^2 \delta P - 4\pi Ga\overline{\rho}\delta \nonumber
	\\
	&\Rightarrow-a\ddot{\delta} + H.(-a\dot{\delta}) = -\frac{1}{a\overline{\rho}}c_s^2\overrightarrow{\nabla}^2.\delta \rho - 4 \pi Ga\overline{\rho}\delta \nonumber
	\\
	&\Rightarrow \ddot{\delta} + H\dot{\delta} = \frac{c_s^2}{a^2}\overrightarrow{\nabla}^2\delta + 4\pi Ga\overline{\rho}\delta \nonumber
	\\
	&\boxed{\Rightarrow \ddot{\delta}+H\dot{\delta}-\frac{c_s^2}{a^2}\overrightarrow{\nabla}^2\delta = 4\pi Ga\overline{\rho}\delta} \label{einstein de sitter: 1}
	\end{align}
	
	During matter-dominated era, \eqref{einstein de sitter: 1} reads
	\begin{equation}\label{key}
	\ddot{\delta_m} + 2H\dot{\delta_m} - 4\pi G\overline{\rho}_m\delta_m = 0
	\end{equation}
	Since $ a\propto t^{\frac{2}{3}} $, $ H=\dfrac{3}{2}t $. So we have
	\begin{equation}\label{key}
	\ddot{\delta_m} + \frac{4}{3t}\dot{\delta_m} - \frac{2}{3t^2}\delta_m = 0
	\end{equation}
	Here we have used the fact that in the matter dominated era, $ 4\pi G \overline{\rho} = \dfrac{3}{2}H^2 $.
	\\
	To solve this equation, we assume a solution of the form $ \delta_m = kt^p $. \\ \\
	\begin{center}
		Then, $ \dot{\delta_m} = pkt^{p-1} $ and $ \ddot{\delta_m} = p(p-1)kt^{p-2}$.
	\end{center}
	
	Substituting, we get
	
	\begin{align}\label{key}
	&p(p-1)kt^{p-2} + \frac{4}{3}pkt^{p-2} - \frac{2}{3}kt^{p-2} = 0 \nonumber
	\\
	&\Rightarrow p(p-1)+\frac{4p}{3} - \frac{2}{3} = 0 \nonumber
	\\
	&\Rightarrow 3p^2 - 3p + 4p -2 =0 \nonumber
	\\
	&\Rightarrow 3p^2+p-2 = 0 \nonumber
	\\
	&\Rightarrow p=-1 \text{ or } p=\frac{2}{3} \nonumber
	\end{align}
	
	\begin{equation}\label{key}
	\therefore \delta_m \propto \begin{cases}t^{-1}, &\propto a^{-\frac{3}{2}}\\
	t^{\frac{2}{3}}, &\propto a
	\end{cases}
	\end{equation}
	
	Hence, the growing mode of fluctuations grows like the scale factor during the matter-dominated era.
\end{document}